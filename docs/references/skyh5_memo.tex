\documentclass[11pt, oneside]{article}
\usepackage{geometry}
\geometry{letterpaper}
\usepackage{graphicx}
\usepackage[titletoc,toc,title]{appendix}
\usepackage{amssymb}
\usepackage{physics}
\usepackage{array}
\usepackage{makecell}

\usepackage{hyperref}
\hypersetup{
    colorlinks = true
}

\usepackage{cleveref}

\title{Memo: SkyH5 file format}
\author{Bryna Hazelton, and the pyradiosky team}
\date{October 5, 2023}

\begin{document}
\maketitle
\tableofcontents
\section{Introduction}
\label{sec:intro}

This memo introduces a new HDF5\footnote{\url{https://www.hdfgroup.org/}}-based
file format of a SkyModel object in \verb+pyradiosky+\footnote{\url{https://github.com/RadioAstronomySoftwareGroup/pyradiosky}},
a python package that provides objects and interfaces for representing diffuse, extended and compact astrophysical radio sources.
Here, we describe the required and optional elements and the structure of this file format, called \textit{SkyH5}.

We assume that the user has a working knowledge of HDF5 and the associated
python bindings in the package \verb+h5py+\footnote{\url{https://www.h5py.org/}}, as
well as SkyModel objects in pyradiosky. For more information about HDF5, please
visit \url{https://portal.hdfgroup.org/display/HDF5/HDF5}. For more information
about the parameters present in a SkyModel object, please visit
\url{https://pyradiosky.readthedocs.io/en/latest/skymodel.html}.
Examples of how to interact with SkyModel objects in pyradiosky are available at
\url{http://pyradiosky.readthedocs.io/en/latest/tutorial.html}.

Note that throughout the documentation, we assume a row-major convention (i.e.,
C-ordering) for the dimension specification of multi-dimensional arrays. For
example, for a two-dimensional array with shape ($N$, $M$), the $M$-dimension is
varying fastest, and is contiguous in memory. This convention is the same as
Python and the underlying C-based HDF5 library. Users of languages with the
opposite column-major convention (i.e., Fortran-ordering, seen also in MATLAB
and Julia) must transpose these axes.

\section{Overview}
\label{sec:overview}
A SkyH5 object contains data representing catalogs and maps of
astrophysical radio sources, including the associated metadata necessary to interpret them.
A SkyH5 file contains two primary HDF5 groups: the \verb+Header+ group, which contains the metadata, and
the \verb+Data+ group, which contains the Stokes parameters representing the
flux densities or temperatures of the sources. Datasets in the \verb+Data+ group
are can be passed through HDF5's compression
pipeline, to reduce the amount of on-disk space required to store the data.
However, because HDF5 is aware of any compression applied to a dataset, there is
little that the user has to explicitly do when reading data. For users
interested in creating new files, the use of compression is optional in the
SkyH5 format, because the HDF5 file is self-documenting in this regard.

Many of the datasets in SkyH5 files have units associated with them (represented as astropy Quantity objects on the SkyModel object).
The units are stored as attributes on the datasets with the name `unit'.
Datasets that derive from other astropy objects (e.g. astropy.time.Time,
astropy.coordinate.EarthLocation, astropy.coordinate.Latitude, astropy.coordinate.Longitude)
also have an `object\_type' attribute indicating the object type.

In the discussion below, we discuss required and optional datasets in the
various groups. We note in parenthesis the corresponding attribute of a SkyModels
object. Note that in nearly all cases, the names are coincident, to make things
as transparent as possible to the user.

\section{Header}
\label{sec:header}
The \verb+Header+ group of the file contains the metadata necessary to interpret
the data. We begin with the required parameters, then continue to optional
ones. Unless otherwise noted, all datasets are scalars (i.e., not arrays). The
precision of the data type is also not specified as part of the format, because
in general the user is free to set it according to the desired use case (and
HDF5 records the precision and endianness when generating datasets). When using
the standard \verb+h5py+-based implementation in pyradiosky, this typically
results in 32-bit integers and double precision floating point numbers. Each
entry in the list contains \textbf{(1)} the exact name of the dataset in the
HDF5 file, in boldface, \textbf{(2)} the expected datatype of the dataset, in
italics, \textbf{(3)} a brief description of the data, and \textbf{(4)} the name
of the corresponding attribute on a SkyModel object.

Note that string datatypes should be handled with care. See
the Appendix in the UVH5 memo (\url{https://github.com/RadioAstronomySoftwareGroup/pyuvdata/blob/main/docs/references/uvh5_memo.pdf})
for appropriately defining them for interoperability between different HDF5 implementations.

\subsection{Required Parameters}
\label{sec:req_params}
\begin{itemize}

\item \textbf{component\_type}: \textit{string} The type of components in the SkyModel. The options are: `healpix' and `point'.
If component_type is `healpix', the components are the pixels in a HEALPix map in units compatible with K or Jy/sr.
If the component_type is `point', the components are point-like sources, or point like components of extended sources,
in units compatible with Jy or K sr. Some additional parameters are required depending on the component type. (\textit{component\_type})

\item \textbf{Ncomponents}: \textit{int} The number of components in the SkyModel. This can be the number of individual
compact sources, or it can include components of extended sources, or the number of pixels in a map. (\textit{Ncomponents})

\item \textbf{spectral\_type}: \textit{string} This describes the type of spectral model for the components. The options are:
`spectral\_index', `subband', `flat', or `full'. If the spectral model uses a spectral index, the `reference\_frequency' and
`spectral\_index` parameters are required. The convention for the spectral index is $I=I_0 \frac{f}{f_0}^{\alpha}$, where
$I_0} is the `stokes` parameter at the `reference\_frequency' parameter $f_0$ and $\alpha$ is the `spectral\_index` parameter.
Note that the spectral index is assumed to apply in the units of the stokes parameter (i.e. there is no additive factor of 2 applied
to convert between temperature and flux density units).
The subband spectral model is used for catalogs with multiple flux measurements at different frequencies (i.e. GLEAM
\url{https://www.mwatelescope.org/science/galactic-science/gleam/}). For subband  spectral models, the `freq_array`
and `freq_edge_array` parameters are required to give the nominal (usually the central) frequency and the top and bottom of
each subband respectively.
The flat spectral model assumes no spectral flux dependence, which can be useful for testing.
The full spectral model is used for catalogs with flux values at multiple frequencies that are not expected to have flux correlations as a function of frequency, so cannot not be interpolated to frequencies not included in the catalog. This is a good representation for e.g. Epoch of Reionization signal cubes.
For full  spectral models, the `freq_array` parameter is required to give the frequencies.
(\textit{spectral\_type})

\item \textbf{Nfreqs}: \textit{int}
Number of frequencies if spectral_type is `full' or `subband', 1 otherwise. (\textit{Nfreqs})

\item \textbf{history}: \textit{string} The history of the catalog. (\textit{history})



\subsection{Optional Parameters}
\label{sec:opt_params}
\begin{itemize}
\item \textbf{name}: \textit{string} The name for each component. This is a one-dimensional array of size (Ncomponents).
Note this is \textbf{required} if the component\_type is `point'. (\textit{name})

\item \textbf{skycoord}:
A nested dataset that contains the information to create an \verb+astropy.coordinates.SkyCoord+object representing the component positions.
Note this is \textbf{required} if the component\_type is `point'. The keys must include:
	\begin{itemize}
	\item \textbf{frame}: \textit{string} The name of the coordinate frame (e.g. `icrs', `fk5', `galactic'). Must be a frame supported by \verb+astropy+.
	\item \textbf{representation\_type}: \textit{string} The representation type, one of `spherical', `cartesian' or `cylindrical'
	\item Representation component names (e.g. \textbf{ra}, \textbf{dec}, \textbf{alt}, \textbf{az}): \textit{float} Two or three such components must be present, 	which ones are required depend on the frame. These are one-dimensional arrays of size (Ncomponents).
And may include any other attributes accepted as input parameters for an \verb+astropy.coordinates.SkyCoord+object (e.g. `obstime', `equinox', `location').
Each of these datasets may have `unit' and `object\_type' attributes and may be either a scalar or a one-dimensional array of size (Ncomponents) as appropriate.
(\textit{skycoord})

\item \textbf{nside}: \textit{int}
The HEALPix nside parameter. Note this is \textbf{required} if the component\_type is `healpix' and should not be defined otherwise. (\textit{nside})

\item \textbf{hpx\_order}: \textit{string}
The HEALPix pixel ordering convention, either `ring' or `nested'.
Note this is \textbf{required} if the component\_type is `healpix' and should not be defined otherwise. (\textit{hpx\_order})

\item \textbf{hpx\_frame}:  A nested dataset that contains the information to describe an astropy coordinate frame giving the HEALPix coordinate frame.
This is similar to the skycoord dataset described above but it does not contain the representation\_type or the representation component names.
Note this is \textbf{required} if the component\_type is `healpix' and should not be defined otherwise.
The keys must include:
	\begin{itemize}
	\item \textbf{frame}: \textit{string} The name of the coordinate frame (e.g. `icrs', `fk5', `galactic'). Must be a frame supported by \verb+astropy+.
And may include any other scalar attributes accepted as input parameters for an \verb+astropy.coordinates.SkyCoord+object (e.g. `obstime', `equinox', `location').
Each of these datasets may have `unit' and `object\_type' attributes as appropriate.
(\textit{hpx\_frame})

\item \textbf{hpx\_inds}: \textit{int}
The HEALPix indices for the included components. Does not need to include all the HEALPix pixels in the map.
This is a one-dimensional array of shape (Ncomponents,).
Note this is \textbf{required} if the component\_type is `healpix' and should not be defined otherwise. (\textit{hpx\_inds})

\item \textbf{freq\_array}: \textit{float}
Frequency array giving the nominal (or central) frequency in a unit that can be converted to Hz.
Note this is \textbf{required} if the spectral\_type is `full' or `subband' and should not be defined otherwise. (\textit{freq\_array})

\item \textbf{freq\_edge\_array}: \textit{float}
Array giving the frequency band edges in a unit that can be converted to Hz, only required if spectral_type is `subband'.
This is a two dimensional array with shape (2, Nfreqs). The zeroth index in the first dimension holds the lower band edge
and the first index holds the upper band edge.
Note this is \textbf{required} if the spectral\_type is `subband' and should not be defined otherwise. (\textit{freq\_edge\_array})
